\chapter{صفحات پایانی}
\section{واژه‌نامه با زیندی}
برای تولید واژه‌نامه با زیندی قبل از هر کار لازم است زیندی تحت ویندوز را نصب کنید.
در ابتدا بسته‌ی {\lr{glossaries}} را با {\lr{option}}، {\lr{Xindy}} فراخوانی کنید. در مرحله بعد دو استایل برای واژه‌نامه ها با دستور {\lr{newglossarystyle}} تعریف نموده ایم. یکی برای واژه نامه فارسی به انگلیسی یکی هم برای انگلیسی به فارسی. 

در مرحله سوم دو نوع واژه نامه بادستور {\lr{newglossary}} تعریف می کنیم. دقت کنید با این کار ۵ فایل با پسوند {\lr{blo,glo,gls,glo,glg}} تولید می شود. من سه حالت برای وارد کردن واژه ها در واژه نامه تعریف کردم.
\begin{itemize}
\item {\lr{inpdic}}:
این دستور واژه ها را هم در واژه‌نامه وارد می‌کند و هم در پاورقی می‌آورد و خود واژه را در متن نیز قرار می‌دهد. مثل: {\inpdic{همتافتن}{Multiplex}}

\item {\lr{indic}}:
همانند {\lr{inpdic}} است، تنها ترجمه واژه در پاورقی نمی آید. مثل: {\indic{همتافتگر}{Multiplexer}}
\item {\lr{ingls}}:
این دستور باعث می‌شود تنها واژه در واژه نامه ظاهر شود و اصلا در متن ظاهر نمی شود. مثل: {\ingls{همتافتگری}{Multiplexing}}. همان طور که می بینید در این مثال کلمه (همتافتگری) تنها در واژه نامه آمده است و اصلا در متن ظاهر نشده است. 
\end{itemize}

مهم ترین مرحله کامپایل برنامه است که باید به صورت دنباله زیر باشد: (این تنظیمات برای {\lr{texmaker}} است.)
\begin{itemize}
\begin{LTRitems}
\item
\verb+ xelatex -interaction=nonstopmode -synctex=-1 %.tex+
\item
\verb+ xindy -L persian -C utf8 -I xindy -M %.xdy -t %.glg -o %.gls %.glo +
\item
\verb+ xindy -L persian -C utf8 -I xindy -M %.xdy -t %.blg -o %.bls %.blo +
\item
\verb+ xelatex -interaction=nonstopmode -synctex=-1 %.tex+
\end{LTRitems}
\end{itemize}
%\متن‌سیاه{نکته:}
قبل از کپی کردن این دستورها در تک‌میکر برای بردن به پنجره‌ی \lr{Command Promp}، انتخاب‌شان کنید و روی‌شان کلیک راست کنید و گزینه ی \lr{Remove  Unicode Control Characters...} را بزنید. 

دقت کنید که مورد دوم در {\lr{Bidi Texmaker}}  آمده است، ولی مورد سوم وجود ندارد، و باید به صورت دستی وارد کنید. یعنی در  {\lr{User Command}} آن را تعریف کنیم. دقت کنید اگر مورد سوم را انجام ندهید یکی از واژه نامه ها اصلا تولید نمی شود. 

مثال‌هایی دیگر:


\inpdic{دسترسی چندگانه}{Multiple Access}
\inpdic{فراگردی}{Roaming}
\inpdic{واگذاری}{Handover}% FA: , واگذاری Comm: دگرسپاری
\inpdic{جایگشت}{Handoff}
\ingls{هزینه}{Charging}
\ingls{افزونگی}{Redundancy}
\ingls{شیار زمانی}{Time Slot}
\ingls{گذردهی}{Throughput}
\ingls{اشتراک زمانی}{Time Sharing}
\ingls{زمان‌بندی}{Timing}
\ingls{نمونه}{Sample}
\ingls{نمونه‌برداری}{Sampling}
\ingls{درهم‌ساختن}{Scramble}
\ingls{درهم‌ساز}{Scrambler}
\indic{کددرهم‌ساز}{Srambling Code}
\indic{خدمت}{Service}،
\indic{پهنای باند}{Bandwidth}،
\indic{باند‌پایه}{Base Band}،
\indic{دودویی}{Binary}،


\section{مراجع}
برای مشاهده قالب‌بندی مربوط به مراجه می‌توانید مراجع این پایان‌نامه نمونه را ملاحظه کنید.
مرجع \cite{1} یک مقاله فارسی چاپ شده در مجله، \cite{2} یک کتاب فارسی، \cite{3} یک مقاله کنفرانسی داخلی، \cite{4} یک پایان‌نامه ارشد فارسی، \cite{5}  یک پایان‌نامه دکتری فارسی، \cite{6} یک منبع اینترنتی فارسی(متفرقه)،  \cite{7} یک مقاله انگلیسی چاپی، \cite{8} یک مقاله انگلیسی الکترونیکی، \cite{9} یک کتاب انگلیسی، \cite{10} یک مقاله کنفرانسی خارجی، \cite{11} یک پایان‌نامه ارشد انگلیسی، \cite{12} پایان‌نامه دکتری انگلیسی و \cite{13} یک مقاله انگلیسی از یک مجموعه مقالات  است.



